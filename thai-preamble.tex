%% ---- Preamble Injected from thaipdf package (BEGIN) ---- %%

%% ขอบคุณแหล่งข้อมูลต่อไปนี้ สำหรับคำแนะนำการตั้งค่าภาษาไทยใน LaTeX %%
%   ฑิตยา หวานวารี: http://pioneer.netserv.chula.ac.th/~wdittaya/LaTeX/LaTeXThai.pdf
%   mathmd from Github; https://github.com/mathmd/polygloTeX

%% ---------------- เริ่มการตั้งค่าภาษาไทย ---------------- %%

%% ---- ตั้งค่าให้ตัดคำภาษาไทย ---- %%
\XeTeXlinebreaklocale "th"
\XeTeXlinebreakskip = 0pt plus 0pt % เพิ่มความกว้างเว้นวรรคให้ความยาวแต่ละบรรทัดเท่ากัน

%% ---- font settings ---- %%
\usepackage{fontspec} % For Thai font
\defaultfontfeatures{Mapping=tex-text} % map LaTeX formating, e.g., ``'', to match the current font

% To change the main font, uncomment one of the below command, and make sure you have these fonts installed.
% \setmainfont{TeX Gyre Termes} % Free Times
% \setsansfont{TeX Gyre Heros} % Free Helvetica
% \setmonofont{TeX Gyre Cursor} % Free Courier

% ตั้งฟอนต์หลักภาษาไทย ที่น่าใช้ เช่น: "TH Sarabun New", "Laksaman"
\newfontfamily{\thaifont}[Scale=MatchUppercase,Mapping=textext]{TH
Sarabun New}
\newenvironment{thailang}{\thaifont}{} % create environment for Thai language
\usepackage[Latin,Thai]{ucharclasses} % ตั้งค่าให้ใช้ "thailang" environment เฉพาะ string ที่เป็น Unicode ภาษาไทย

\setTransitionTo{Thai}{\begin{thailang}} % Use environment "thailang" when found Thai font.
\setTransitionFrom{Thai}{\end{thailang}} % If not found, end the environment.

%% ---- Spacing ---- %%
\renewcommand{\baselinestretch}{1.5} % Line Spacing (1.5 is recommended for Thai language)


%% ---- using alphabatic language ---- %%
\usepackage{polyglossia}
\setdefaultlanguage{english} % it is preferrable to set English as the main language, since the numeric system is compatible with most LaTeX features such as 'enumerate' and so on
\setotherlanguages{thai}

\AtBeginDocument\captionsthai % allow captions to be in Thai

%% ---------------- จบการตั้งค่าภาษาไทย! ---------------- %%
%% ---------------- สามารถใส่การตั้งค่าอื่นๆ เพิ่มเติมได้ ---------------- %%

%% ---- hyperref settings (uncomment for colorful link) ---- %%
% \usepackage{hyperref}
% \usepackage{url}
% \usepackage{cite}
% \usepackage{xcolor}
% \hypersetup{
%     colorlinks, % for colorful link
%     linkcolor={red!50!black},
%     citecolor={blue!50!black},
%     urlcolor={blue!80!black}
%     }

%% ---- Preamble Injected from thaipdf package (END) ---- %%
